\documentclass[12pt]{article}

\usepackage{graphicx}
\usepackage{amsfonts}
\usepackage{amsmath}
\usepackage{amsthm}
\usepackage{verbatim}
\usepackage[utf8]{inputenc}


\topmargin=-1cm
\leftmargin=0cm
%\rightmargin=0cm
\oddsidemargin=0.0cm
\evensidemargin=2cm
\textwidth=16.5cm
\textheight=22cm

\newcommand{\bfx}{\mbox{\boldmath $x$}}
\newcommand{\bfy}{\mbox{\boldmath $y$}}
\newcommand{\bfz}{\mbox{\boldmath $z$}}
\newcommand{\bfv}{\mbox{\boldmath $v$}}
\newcommand{\bfu}{\mbox{\boldmath $u$}}
\newcommand{\bfF}{\mbox{\boldmath $F$}}
\newcommand{\bfJ}{\mbox{\boldmath $J$}}
\newcommand{\bfg}{\mbox{\boldmath $g$}}
\newcommand{\bfc}{\mbox{\boldmath $c$}}
\newcommand{\bfxi}{\mbox{\boldmath $\xi$}}
\def\be{\begin{equation}}
\def\ee{\end{equation}}
\def\ba{\begin{array}}
\def\ea{\end{array}}
\def\d{\mbox{d}}




\pagestyle{empty}

\begin{document}

\begin{center}
{\huge Hermes1D notes}\\
\vspace{0.6cm}
Ondřej Čertík, Pavel Šolín\\
$hp$-FEM group, University of Nevada, Reno\\
http://hpfem.math.unr.edu\\
March 2009
\end{center}
\begin{abstract}
Some notes to our Hermes 1D code.
\end{abstract}

\section{Introduction}

A lots of (but not all) ODEs can be put into the form
$$
{\d u_1\over\d x}=f_1(u_1, u_2, \dots, u_N, x)
$$
$$
...
$$
$$
{\d u_N\over\d x}=f_N(u_1, u_2, \dots, u_N, x)
$$
We use the notation ${\bf U}\equiv(u_1,\dots,u_N)$, ${\bf
F}\equiv(f_1,\dots,f_N)$:
$${\d {\bf U}\over\d x}={\bf F}({\bf U}, x)$$
We then define ${\bf R}({\bf U})={\d {\bf U}\over\d x}-{\bf F}({\bf U}, x)$, so
we want to find such $u_1$, \dots, $u_N$ that:
$${\bf R}({\bf U})=0$$
Well, actually ${\bf R}$ is more like this:
$$R_i(Y)=\int_0^T{\partial\varphi_j\over\partial x}\varphi_i-
f_i(Y)\varphi_i\d x
$$
This is in general non-linear problem, so we calculate a Jacobian:
$$J(Y)={\partial R_i\over\partial u_j}(Y)=\int_0^T{\partial\varphi_j\over\partial x}\varphi_i-
{\partial f_i\over \partial u_j}(Y)\varphi_j\varphi_i\d x
$$
And then solve iteratively
$$J(Y^n)\ \delta Y^{n+1}=-R(Y^n)$$
where $\delta Y^{n+1} = Y^{n+1}-Y^{n}$.

\end{document}
