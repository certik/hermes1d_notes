\documentclass[12pt]{article}

\usepackage{graphicx}
\usepackage{amsfonts}
\usepackage{amsmath}
\usepackage{amsthm}
\usepackage{verbatim}
\usepackage[utf8]{inputenc}


\topmargin=-1cm
\leftmargin=0cm
%\rightmargin=0cm
\oddsidemargin=0.0cm
\evensidemargin=2cm
\textwidth=16.5cm
\textheight=22cm

\newcommand{\bfx}{\mbox{\boldmath $x$}}
\newcommand{\bfy}{\mbox{\boldmath $y$}}
\newcommand{\bfz}{\mbox{\boldmath $z$}}
\newcommand{\bfv}{\mbox{\boldmath $v$}}
\newcommand{\bfu}{\mbox{\boldmath $u$}}
\newcommand{\bfF}{\mbox{\boldmath $F$}}
\newcommand{\bfJ}{\mbox{\boldmath $J$}}
\newcommand{\bfU}{\mbox{\boldmath $U$}}
\newcommand{\bfY}{\mbox{\boldmath $Y$}}
\newcommand{\bfR}{\mbox{\boldmath $R$}}
\newcommand{\bfg}{\mbox{\boldmath $g$}}
\newcommand{\bfc}{\mbox{\boldmath $c$}}
\newcommand{\bfxi}{\mbox{\boldmath $\xi$}}
\def\be{\begin{equation}}
\def\ee{\end{equation}}
\def\ba{\begin{array}}
\def\ea{\end{array}}
\def\d{\mbox{d}}




\pagestyle{empty}

\begin{document}

\begin{center}
{\huge Hermes1D}\\[2mm]
{\Large an $hp$-FEM solver for ODE}\\
\vspace{0.6cm}
Ondřej Čertík, Pavel Šolín\\
$hp$-FEM group, University of Nevada, Reno\\
http://hpfem.math.unr.edu\\
March 2009
\end{center}

\section{Introduction and motivation}

When one speaks about the numerical solution of ODEs, one usually has in mind
initial value problems for equations of the form

$$
{\d u_1\over\d x}=g_1(u_1, u_2, \dots, u_m, x),
$$
\be\label{one}
\vdots
\ee
$$
{\d u_m\over\d x}=g_m(u_1, u_2, \dots, u_m, x).
$$
These are solved in a finite time interval $(0,T)$ using various time-stepping
methods. There are tons of those and some are quite sophisticated (meaning
multistep, higher-order, adaptive, etc.). But all of them have the following
common shortcomings:
\begin{itemize}
\item We would like to prescribe the initial value at $t = 0$ for some solution
      components and the end-time values at $t = T$ for others. Standard
      time stepping methods do not allow this.
\item Global error control is problematic. One only can regulate the time step
      size locally -- this is something like
      "forward mesh refinement''. But one cannot do "backward mesh refinement''
      or coarsening easily.
\item We would like to prescribe a tolerance for the global error and then
      have the problem solved adaptively until this error tolerance is reached,
      without underresolving or overresolving too much. This is virtually
      impossible with adaptive time stepping methods.
\item Standard time integration methods cannot change their order during the
      computation. For example, an adaptive RK4 method remains 4-order all the
      time. This is an analogy for $h$-refinement in FEM, and obviously it is
      highly inefficient. Correctly, the method should either do small
      low-order steps or large high-order steps to be efficient. We would like
      to see such an analogy of $hp$-refinement in ODE methods.
\item We would like to solve more general ODEs than (\ref{one}).
\end{itemize}
This is why we decided to apply the $hp$-FEM methodology to ODEs and see what happens.

\section{Equations}

We implemented the first version of Hermes1D during one day while returning
from the 2009 SIAM CSE conference. First we considered the form (\ref{one}) but
then we realized that with no extra work we can actually assume a much more
general implicit form

$$
f_1(u_1, u_2, \ldots, u_m, u'_1, u'_2, \ldots, u'_m, x) = 0,
$$
\be\label{two}
\vdots
\ee
$$
f_m(u_1, u_2, \ldots, u_m, u'_1, u'_2, \ldots, u'_m, x) = 0.
$$
Note that (\ref{two}) contains (\ref{one}) as a special case.
In fact, (\ref{two}) can be written shortly as
\be\label{qqq}
\bfF(\bfU, \bfU', x) = 0
\ee
where ${\bfU} = (u_1,\dots,u_m)$ and ${\bfF} = (f_1,\dots,f_m)$.

\subsubsection*{Boundary conditions}

So far, we have considered Dirichlet boundary conditions only, which can be
imposed either at the initial time $t = 0$ or the end-time $t = T$. Exactly one
condition per solution component has to be defined.


\section{$hp$-FEM discretization}

As always, the finite element discretization starts from a weak formulation.
With (\ref{two}), the situation is easy and we have

$$
R_1(\bfY) = \int_0^T f_1(u_1, u_2, \ldots, u_m, u'_1, u'_2, \ldots, u'_m, x)v_1 \, \d t = 0,
$$
\be\label{three}
\vdots
\ee
$$
R_N(\bfY) = \int_0^T f_m(u_1, u_2, \ldots, u_m, u'_1, u'_2, \ldots, u'_m, x)v_N \, \d t = 0.
$$
Here $v_1, v_2, \ldots, v_N$ are all basis functions for all solution
components (we can describe this more accurately if needed).  In the standard
sense, all basis functions corresponding to the solution component $u_i$ are
zero where $u_i$ has a Dirichlet boundary condition.  The vector $\bfY = (y_1,
y_2, \ldots, y_N)$ comprises all unknown coefficients of the finite element
basis functions for all solution components. The meshes for the solution
components $u_1, u_2, \ldots, u_m$ could (more precisely: {\em should}) be
different but for now we assume that they are the same.

\section{Newton's method}

We will drive the residual vector $\bfR = (R_1, R_2, \ldots, R_N)$ to zero
using the Newton's method. For that, we need the Jacobi matrix D$\bfR/$D$\bfY$.

Let $1 \le i, j \le N$.
It is easy to calculate that
$$
\frac{\partial R_i}{\partial y_j}
= \int_0^T \frac{\partial f_{m(i)}}{\partial u_{n(j)}}(u_1, u_2, \ldots, u_m, u'_1, u'_2, \ldots, u'_m, x)v_jv_i
$$
\be\label{newt1}
+ \frac{\partial f_{m(i)}}{\partial u'_{n(j)}}(u_1, u_2, \ldots, u_m, u'_1, u'_2, \ldots, u'_m, x)v'_jv_i \, \d t = 0.
\ee
Here, the function $m(i)$ takes a global index $1 \le i \le N$ and returns the
index of the function $f_{m(i)}$ which is associated with $R_i$. Analogously,
$n(j)$ takes a global index $1 \le j \le N$ and returns the index of the
solution component $u_{n(i)}$ where the basis function $v_j$ belongs to.

The integral in (\ref{newt1}) has two parts because the functions $u_s$ and
$u'_s$ depend on the same solution coefficients.  Do not be confused by the
derivatives with respect to $u'_{n(j)}$ in (\ref{newt1}).  The functions $u_s$
and $u'_s$ are used as independent variables for the differentiation.


\section{Numerical Examples}

The best way to understand the above machinery is to solve examples which we
will do in this section.

\subsection{Classical Harmonic Oscillator}

One of the important equations from the classical mechanics is the harmonic
oscillator equation:
$$u''(x)+u(x)=0$$
and for this example we choose a simple boundary conditions $u(0)=0$ and
$u'(0)=1$ so the solution is $u(x)=\sin
x$. First let's rewrite the equation into the form (\ref{one}):
$$
{\d u_1\over\d x}=g_1(u_1, u_2, x)=u_2
$$
$$
{\d u_2\over\d x}=g_2(u_1, u_2, x)=-u_1
$$
where $u_1=u$ is the function we seek and $u_2=u'$ is its derivative.
Then let's write it in the form (\ref{two}):
$$
f_1(u_1, u_2, u'_1, u'_2, x) = g_1(u_1, u_2, x)-u_1'=-u_1'+u_2 = 0,
$$
$$
f_2(u_1, u_2, u'_1, u'_2, x) = g_2(u_1, u_2, x)-u_2'=-u_2'-u_1 = 0,
$$
and (\ref{qqq}):
$$
\bfF(\bfU, \bfU', x) = 0
$$
where ${\bfU} = (u_1, u_2)$ and ${\bfF} = (f_1, f_2)=(-u_1'+u_2, -u_2'-u_1)$.
The weak formulation is:
$$
R_1(\bfY) = \int_0^T f_1(u_1, u_2, u'_1, u'_2, x)v_1 \, \d t =
\int_0^T (-u_1'+u_2)v_1 \, \d t
=0,
$$
$$
R_2(\bfY) = \int_0^T f_2(u_1, u_2, u'_1, u'_2, x)v_1 \, \d t =
\int_0^T (-u_2'-u_1)v_2 \, \d t
=0,
$$ To evaluate the Jacobi matrix D$\bfR/$D$\bfY$ for the Newton's iteration, we need the following
Jacobians:
$$
\left({{\rm D}\bfF\over{\rm D}\bfU}\right)_{mn}=
\frac{\partial f_m}{\partial u_n}(u_1, u_2, u'_1, u'_2, x)
=
\left( \begin{array}{c}
-u_1'+u2 \\
-u_2'-u1 \\
\end{array} \right)
\left( \begin{array}{cc}
\overleftarrow{\partial_{u_1}} & \overleftarrow{\partial_{u_2}} \\
\end{array} \right)
=
\left( \begin{array}{cc}
0 & 1 \\
-1 & 0 \\
\end{array} \right)
$$
$$
\left({{\rm D}\bfF\over{\rm D}\bfU'}\right)_{mn}=
\frac{\partial f_m}{\partial u'_n}(u_1, u_2, u'_1, u'_2, x)
=
\left( \begin{array}{c}
-u_1'+u2 \\
-u_2'-u1 \\
\end{array} \right)
\left( \begin{array}{cc}
\overleftarrow{\partial_{u_1'}} & \overleftarrow{\partial_{u_2'}} \\
\end{array} \right)
=
\left( \begin{array}{cc}
-1 & 0 \\
0 & -1 \\
\end{array} \right)
$$
where $\overleftarrow{\partial_{u_1}}$ is a partial derivative with respect to
$u_1$ but acting to the left.

To solve this problem with Hermes, all we have to do is to specify the
following information:
$${\bfF} =
\left( \begin{array}{c}
-u_1'+u2 \\
-u_2'-u1 \\
\end{array} \right)
$$
$$
{{\rm D}\bfF\over{\rm D}\bfU}=
\left( \begin{array}{cc}
0 & 1 \\
-1 & 0 \\
\end{array} \right)
$$
$$
{{\rm D}\bfF\over{\rm D}\bfU'}=
\left( \begin{array}{cc}
-1 & 0 \\
0 & -1 \\
\end{array} \right)
$$
$$u_1(0)=0$$
$$u_2(0)=1$$
and Hermes will solve for $\bfF=0$. This is implemented in
\texttt{examples/sin.py}.

\subsection{Quantum Harmonic Oscillator}

The corresponding quantum mechanics problem to the previous one is the quantum
harmonic oscillator for one particle in 1D:
$$
i\hbar{\partial\over\partial t}\psi(x, t)=
-{\hbar^2\over2m}{\partial^2\over\partial x^2}\psi(x,t)+V(x)\psi(x,t)
$$
$$
V(x)={1\over2}m\omega^2x^2
$$
This is a partial differential equation for the time evolution of the wave
function $\psi(x, t)$, but one method to solve it is the
eigenvalues expansion:
$$\psi(x,t) = \sum_E c_E\psi_E(x)e^{-{i\over\hbar}Et}$$
where the sum goes over the whole spectrum (for continuous spectrum the sum
turns into an integral), the $c_E$ coefficients are determined from the initial condition
and $\psi_E(x)$ satisfies the one dimensional one particle time independent
Schr\"odinger equation:
$$
-{\hbar^2\over2m}{\d^2\over\d x^2}\psi_E(x)+V(x)\psi_E(x)=E\psi_E(x)
$$
and this is just an ODE and thus can be solved with Hermes1D. There can be many
types of boundary conditions for this equation, depending on the physical
problem, but in our case we simply have $\lim_{x\to\pm\infty}\psi_E(x)=0$ and
the normalization condition $\int_{-\infty}^\infty|\psi_E(x)|^2\d x=1$.

We can set $m=\hbar=1$ and from now on we'll just write $\psi(x)$ instead of
$\psi_E(x)$:
$$
-{1\over2}{\d^2\over\d x^2}\psi(x)+V(x)\psi(x)=E\psi(x)
$$
and we will solve it on the interval $(a, b)$ with the boundary condition
$\psi(a)=\psi(b)=0$. The weak formulation is
$$
\int_a^b{1\over2}{\d\psi(x)\over\d x}{\d v(x)\over\d x}+V(x)\psi(x)v(x)\,\d x
-\left[{\d\psi(x)\over\d x}v(x)\right]^a_b
=E\int_a^b\psi(x)v(x)\,\d x
$$
but due to the boundary condition $v(a)=v(b)=0$ so
$\left[\psi'(x)v(x)\right]^a_b=0$ and we get
$$
\int_a^b{1\over2}{\d\psi(x)\over\d x}{\d v(x)\over\d x}+V(x)\psi(x)v(x)\,\d x
=E\int_a^b\psi(x)v(x)\,\d x
$$
And the finite element formulation is then $\psi(x)=\sum_j y_j\phi_j(x)$ and
$v=\phi_i(x)$:
$$
\left(\int_a^b{1\over2}\phi_i'(x)\phi_j'(x)+V(x)\phi_i(x)\phi_j(x)\,\d x\right)
y_j
=E\int_a^b\phi_i(x)\phi_j(x)\,\d x\ y_j
$$
which is a generalized eigenvalue problem:
$$
A_{ij}y_j=EB_{ij}y_j
$$
with
$$
A_{ij}=\int_a^b{1\over2}\phi_i'(x)\phi_j'(x)+V(x)\phi_i(x)\phi_j(x)\,\d x
$$
$$
B_{ij}=\int_a^b\phi_i(x)\phi_j(x)\,\d x
$$


\subsection{Radial Schr\"odinger Equation}

Another important example is the three dimensional one particle time
independent Schr\"odinger equation for a spherically symmetric potential:
$$
-{1\over2}\nabla^2\psi({\bf x})+V(r)\psi({\bf x})=E\psi({\bf x})
$$
The way to solve it is to separate the equation into radial and angular parts
by writing the Laplace operator in spherical coordinates as:
$$
\nabla^2f = 
{\partial^2 f\over\partial\rho^2}
+{2\over \rho}{\partial^2 f\over\partial\rho^2}
-{L^2\over \rho^2}
$$
$$
L^2=
-{\partial^2 f\over\partial\theta^2}
-{1\over\sin^2\theta}{\partial^2 f\over\partial\phi^2}
-{1\over\tan\theta}{\partial f\over\partial\theta}
$$
Substituting $\psi=R(\rho)Y(\theta,\phi)$ into the Schr\"odinger equation
yields:
$$-{1\over2}\nabla^2(RY)+VRY=ERY$$
$$-{1\over2}R''Y-{1\over\rho}R'Y+{L^2RY\over2\rho^2}+VRY=ERY$$
Using the fact that $L^2Y=l(l+1)Y$ we can cancel $Y$ and we get the radial
Schr\"odinger equation:
$$-{1\over2}R''-{1\over\rho}R'+{l(l+1)R\over2\rho^2}+VR=ER$$
The solution is then:
$$\psi({\bf x})=\sum_{nlm}c_{nlm}R_{nl}(r)Y_{lm}\left({\bf x}\over r\right)$$
where $R_{nl}(r)$ satisfies the radial Schr\"odinger equation (from now on we
just write $R(r)$):
$$-{1\over2}R''(r)-{1\over r}R'(r)+\left(V+{l(l+1)\over2r^2}\right)R(r)=ER(r)$$
Again there are many types of boundary conditions, but the most common case is
$\lim_{r\to\infty}R(r)=0$ and $R(0)=1$ or $R(0)=0$. One solves this equation on
the interval $(0, a)$ for large enough $a$.

The procedure is similar to the previous example, only we need to remember that
we always have to use covariant integration (in the previous example the
covariant integration was the same as the coordinate integration),
in this case $r^2\sin\theta \d
r\d\theta\d\phi$, so the weak formulation is:
$$\int \left(-{1\over2}R''(r)-{1\over
r}R'(r)+\left(V+{l(l+1)\over2r^2}\right)R(r)\right)v(r)r^2\sin\theta \d
r\d\theta\d\phi=$$
$$
=\int ER(r) v(r)r^2\sin\theta \d r\d\theta\d\phi$$
Integrating over the angles gives $4\pi$ which we cancel out at both sides and
we get:
$$\int_0^a \left(-{1\over2}R''(r)-{1\over
r}R'(r)+\left(V+{l(l+1)\over2r^2}\right)R(r)\right)v(r)r^2 \d r=$$
$$
=E\int_0^a R(r) v(r)r^2 \d r$$
We apply per partes to the first two terms on the left hand side:
$$\int_0^a \left(-{1\over2}R''(r)-{1\over r}R'(r)\right)v(r)r^2 \d r
=\int_0^a -{1\over2r^2}\left(r^2 R'(r)\right)'v(r)r^2 \d r=
$$
$$
=\int_0^a -{1\over2}\left(r^2 R'(r)\right)'v(r) \d r
=\int_0^a {1\over2}r^2 R'(r)v'(r) \d r-[r^2R'(r)v(r)]_0^a=
$$
$$
=\int_0^a {1\over2} R'(r)v'(r) r^2\d r
$$
We used the fact that $R'(a)=0$ and $R'(0)=\rm const.$, so the boundary term vanishes. The weak formulation is then:
$$\int_0^a {1\over2}R'(r)v'(r)r^2+
\left(V+{l(l+1)\over2r^2}\right)R(r)v(r)r^2\,\d r
=
E\int_0^aR(r)v(r)r^2\,\d r
$$
or
$$\int_0^a {1\over2}R'(r)v'(r)r^2+
V(r)R(r)v(r)r^2+{l(l+1)\over2} R(r)v(r)\,\d r
=
E\int_0^aR(r)v(r)r^2\,\d r
$$

Another (equivalent) approach is to write a weak formulation for
the 3D problem in cartesian coordinates:
$$
\int_\Omega{1\over2}\nabla\psi({\bf x})\nabla v({\bf x})+V(r)\psi({\bf x})v({\bf
x})\,\d^3 x
=E\int_\Omega\psi({\bf x})v({\bf x})\,\d^3 x
$$
and only then transform to spherical coordinates:
$$
\int_0^{2\pi}\d\varphi\int_0^\pi\d\theta\int_0^a\d r
\left({1\over2}\nabla\psi({\bf x})\nabla v({\bf x})+V(r)\psi({\bf
x})v({\bf x})\right)r^2\sin\theta=
$$
$$
=
E\int_0^{2\pi}\d\varphi\int_0^\pi\d\theta\int_0^a\d r\,
\psi({\bf x})v({\bf x})r^2\sin\theta
$$
The 3d eigenvectors $\psi({\bf x})$ however are not spherically symmetric.
Nevertheless we can still proceed by choosing our basis as
$$v_{ilm}({\bf x})=\phi_{il}(r)Y_{lm}(\theta, \varphi)$$
and seek our solution as
$$\psi({\bf x})=\sum_{jlm}y_{jlm}\phi_{jl}(r)Y_{lm}(\theta, \varphi)$$
Using the properties of spherical harmonics and the gradient:
$$\int Y_{lm} Y_{l'm'} \sin\theta\,\d\theta\,\d\varphi=
\delta_{ll'}\delta_{mm'}$$
$$\int r^2\nabla Y_{lm} \nabla Y_{l'm'} \sin\theta\,\d\theta\,\d\varphi=
l(l+1)\delta_{ll'}\delta_{mm'}$$
$$\nabla f = {\partial f\over \partial r}\boldsymbol{\hat r} + {1\over r}
{\partial f\over\partial\theta}\boldsymbol{\hat\theta}+{1\over r\sin\theta}
{\partial f\over\partial\phi}\boldsymbol{\hat\phi}$$
the weak formulation becomes:
$$
\left(\int_0^a
{1\over2}r^2\phi_{il}'(r)\phi_{jl}'(r)+
{1\over2}X+
{l(l+1)\over2}\phi_{il}(r)\phi_{jl}(r)+
r^2V(r)\phi_{il}(r)\phi_{jl}(r)\,\d r\right)y_{jlm}=
$$
$$ = E\int_0^ar^2 \phi_{il}(r)\phi_{jl}(r)\,\d r\ y_{jlm} $$
where both $l$ and $m$ indices are given by the indices of the particular base
function $v_{ilm}$. The $X$ term is (schematically):
$$X=\int r^2\sin\theta(r)Y_{lm}(\theta,\varphi)
(\phi_{il}\nabla\phi_{jl}+\nabla\phi_{il}\phi_{jl})
\nabla Y_{lm}$$
There is an interesting identity:
$$\int r{\bf \hat r} Y_{lm} \nabla Y_{l'm'} \sin\theta\,\d\theta\,\d\varphi=
0$$
But it cannot be applied, because we have one more $r$ in the expression.
Nevertheless the term is probably zero, as can be seen when we compare the weak
formulation to the one we got directly from the radial equation.

\subsubsection{How Not To Derive The Weak Formulation}

If we forgot that we have to integrate covariantly, this section is devoted
to what happens if we integrate using the coordinate integration. We would get:
$$\int_0^a {1\over2}R'(x)v'(x)-{1\over r}R'(x)v(x)+
\left(V+{l(l+1)\over2r^2}\right)R(x)v(x)\,\d x
=
E\int_0^aR(x)v(x)\,\d x
$$
Notice the matrix on the left hand side is not symmetric. There is another way
of writing the weak formulation by applying per-partes to the $R'(r)v(r)$ term:
$$-\int_0^a{1\over r}R'(x)v(x)\d x=
$$
$$
=\int_0^a{1\over r}R(x)v'(x)\d x
-\int_0^a{1\over r^2}R(x)v(x)\d x
-\left[{1\over r}R'(x)v'(x)\right]_0^a
+\left[{1\over r^2}R'(x)v(x)\right]_0^a
$$
We can use $v(a)=0$ and $R'(a)=0$ to simplify a bit:
$$-\int_0^a{1\over r}R'(x)v(x)\d x=
$$
$$
=\int_0^a{1\over r}R(x)v'(x)\d x
-\int_0^a{1\over r^2}R(x)v(x)\d x
+\lim_{r\to0}\left({R'(x)v'(x)\over r}-{R'(x)v(x)\over r^2}\right)
$$
Since $R(x)\sim r^l$ near $r=0$, we can see that for $l\ge3$ the limits
on the right hand side are zero, but for $l=0, 1, 2$ they are not zero and need
to be taken into account. Let's assume $l\ge3$ for now, then our weak formulation looks like:
$$\int_0^a {1\over2}R'(x)v'(x)+{1\over r}R(x)v'(x)+
\left(V+{l(l+1)\over2r^2}-{1\over r^2}\right)R(x)v(x)\,\d x
=
E\int_0^aR(x)v(x)\,\d x
$$
or
$$\int_0^a {1\over2}R'(x)v'(x)+{1\over r}R(x)v'(x)+
\left(V+{(l-2)(l+1)\over2r^2}\right)R(x)v(x)\,\d x
=
E\int_0^aR(x)v(x)\,\d x
$$
The left hand side is also not symmetric, however we can now take an average of
our both weak formulations to get a symmetric weak formulation:
$$\int_0^a {1\over2}R'(x)v'(x)+{R(x)v'(x)-R'(x)v(x)\over 2r}+
\left(V+{l(l+1)-1\over2r^2}\right)R(x)v(x)\,\d x
=
$$
$$
=
E\int_0^aR(x)v(x)\,\d x
$$
Keep in mind, that this symmetric version is only correct for $l\ge3$. For
$l<3$ we need to use our first nonsymmetric version.

As you can see, this is something very different to what we got in the previous
section. First there were lots of technical difficulties and second the final
result is wrong, since it doesn't correspond to the 3D Schr\"odinger equation.

\section{TODO list}

Currently, the code can handle an arbitrary number of equations and solve them
with elements up to the 10th degree. However, the meshes still have to be the
same for every solution component. The code is not $hp$-adaptive yet. These
things will be fixed as time permits.


\end{document}
